\documentclass[ucs, notheorems, handout]{beamer}

\usetheme[numbers,totalnumbers, nologo]{Statmod}
\usefonttheme[onlymath]{serif}
\setbeamertemplate{navigation symbols}{}

\mode<handout> {
	\usepackage{pgfpages}
	%\setbeameroption{show notes}
	\pgfpagesuselayout{2 on 1}[a4paper, border shrink=5mm]
	\setbeamercolor{note page}{bg=white}
	\setbeamercolor{note title}{bg=gray!10}
	\setbeamercolor{note date}{fg=gray!10}
}

\usepackage[utf8x]{inputenc}
\usepackage[T2A]{fontenc}
\usepackage[russian]{babel}
\usepackage{tikz}
\usepackage{ragged2e}

\usepackage{multirow}
%\usepackage{multirow}
%\usepackage[table,xcdraw]{xcolor}

\newtheorem{theorem}{Теорема}
\newtheorem*{definition}{Определение}
\newtheorem{lemma}{Лемма}
\usepackage{amsmath}
\usepackage{amsthm}
\usepackage{bm}
\usepackage{bbold}

\usepackage{pgfplots}
\pgfplotsset{compat=1.9}

\usepackage{graphicx}
%\usepackage[usenames]{color}
%\usepackage{colortbl}

%\newtheorem{theorem}{Теорерма}
\newtheorem{corollary}[theorem]{Следствие}
%\newtheorem{lemma}[theorem]{Лемма}
\newtheorem{observation}[theorem]{Observation}
\newtheorem{proposition}[theorem]{Предложение (Swanson)}
\newtheorem{proposition2}[theorem]{Предложение}
%\newtheorem{definition}[theorem]{Определение}
\newtheorem{claim}[theorem]{Утверждение}
\newtheorem{fact}[theorem]{Факт}
\newtheorem{assumption}[theorem]{Предположение}
\newtheorem{alg}{Алгоритм}
\newtheorem{zam}{Замечание}

\newtheorem{example}{Пример}[section]

\newenvironment{Proof}{\par\noindent{\bf Доказательство.}}{\hfill$\scriptstyle\blacksquare$}
\newenvironment{ex}{\par\noindent{\bf Пример.}}{}
\newenvironment{pr1}{\par\noindent{\bf Дано:}}{}
\newenvironment{pr2}{\par\noindent{\bf Шаги:}}{}
\newenvironment{pr3}{\par\noindent{\bf Результат:}}{}

\DeclareMathOperator{\tr}{tr}
\DeclareMathOperator{\F}{\mathsf{F}}

\title[Замена непрерывного распределения]{Замена непрерывного распределения на дискретное для применения на практике}
\author[Нагуманова~К. И.]{ Нагуманова Карина Ильнуровна, 19Б.04-мм}

\date{\tiny{Санкт-Петербург\\ 2023г.}}
\institute[Санкт-Петербургский Государственный Университет]{%
	\small
	Санкт-Петербургский государственный университет\\
	Прикладная математика и информатика\\
	\vspace{0.8cm}
	Отчет по преддипломной практике
	%Вычислительная стохастика и статистические модели\\
	\vspace{1.25cm}}
\begin{document}
	
	\begin{frame}
		\titlepage
		\note{Научный руководитель  д.ф.-м.н., доцент Голяндина Н.\,Э.,\\
			кафедра статистического моделирования}
	\end{frame}
	
	\begin{frame}{Введение}
		В практических задачах нередко требуется заменить непрерывное распределение на
		дискретное с сохранением математического ожидания и дисперсии. Одним из методов
		нахождения такого распределения для трехточечной аппроксимации нормального распределения является \textcolor{blue}{\hbox{\textbf{метод Свонсона}}}.
		
		\bigskip
		
		Аппроксимируемые случайные величины складывают и умножают.
		
		\bigskip
		
		\textcolor{red}{\textbf{Задача:}} находить аппроксимацию суммы и произведения логнормальных случайных величин по аппроксимациям исходных случайных величин.
		
		\note{В практических задачах нередко требуется заменить непрерывное распределение на
			дискретное с сохранением математического ожидания и дисперсии. Одним из методов
			нахождения такого распределения для трехточечной аппроксимации нормального распределения является метод Свонсона. Однако в ряде областей, например, в нефтяной промышленности, общепринятым распределением, описывающим запасы нефти, является логнормальное распределение. 
			Соответственно, реальной задачей является аппроксимация логнормального распределения. При этом здесь тоже применим метод Свонсона, потому что логнормальное распределение можно свести к нормальному. Ставится задача нахождения аппроксимации суммы и произведения логнормальных случайных величин по аппроксимациям исходных случайных величин.}
	\end{frame}
	
	
	\begin{frame}{Введение}
		\textbf{План работы.}
		\begin{enumerate}
			\item Рассмотреть общий подход к трехточечной аппроксимации.
			\item Рассмотреть трехточечную аппроксимацию нормального распределения, в целом метод Свонсона и вывод правила 30-40-30.
			\item Рассмотреть трехточечную аппроксимацию логнормального распределения и её  свойства.
			\item Построить алгоритм аппроксимации произведения двух логнормальных распределений.
			\item Построить алгоритм аппроксимации суммы двух логнормальных распределений.
		\end{enumerate}
		
		\note{Имеем следующий план работы. Рассмотреть общий подход к трехточечной аппроксимации, рассмотреть трехточечную аппроксимацию нормального распределения, в целом метод Свонсона и вывод правила 30-40-30, рассмотреть трехточечную аппроксимацию логнормального распределения и её  свойства, построить алгоритм аппроксимации произведения двух логнормальных распределений, построить алгоритм аппроксимации суммы двух логнормальных распределений.}
	\end{frame}
	
	\begin{frame}{Часть 1: Общий подход к трехточечной аппроксимации}
		Пусть $\xi$ "--- непрерывная случайная величина, обозначим  \[m = \mathbf E(\xi), \quad\quad s^{2} = \mathbf D(\xi),\]  $F(x)$ "--- функция распределения,
		$x_{\pi_{1}}$, $x_{\pi_{2}}$, $x_{\pi_{3}}$ "--- квантили $\xi$,
		
		$\tilde{\xi}$ "--- случайная дискретная величина 
		
		\[\tilde{\xi}:\quad\begin{pmatrix} 
			x_{\pi_{1}}&x_{\pi_{2}}&x_{\pi_{3}}\\ 
			p_{1} &  p_{2}  & p_{3}
		\end{pmatrix}\]
		
		\[\tilde{m} = \mathbf E(\tilde{\xi}), \quad\quad \tilde{s}^{2} = \mathbf D(\tilde{\xi}).\]
		\textcolor{red}{\textbf{Задача:}} аппроксимировать распределение случайной величины $\xi$ дискретным распределением $\tilde{\xi}$, то есть найти $p_{1}$, $p_{2}$, $p_{3}$ такие, что 
		\begin{equation*}
			p_{1} + p_{2} + p_{3} = 1, \label{1}
		\end{equation*}
		\begin{equation*}
			\tilde{m} = p_{1}x_{\pi_{1}} + p_{2}x_{\pi_{2}} + p_{3}x_{\pi_{3}} = m, \label{2}
		\end{equation*}
		\begin{equation*}
			\tilde{s}^{2} = p_{1} x_{\pi_{1}}^{2} + p_{2} x_{\pi_{2}}^{2} + p_{3} x_{\pi_{3}}^{2} - m^{2} = s^{2}. \label{3}
		\end{equation*}
		
		\note{Пусть $\xi$ "--- непрерывная случайная величина, вводим основные обозначения для $\xi$, так же вводим дискретную случайную величину, которой будем аппроксимировать $\xi$. Ставится задача аппроксимации распределения $\xi$ дискретным распределением $\tilde{\xi}$, то есть нужно вычислить $p_{1}$, $p_{2}$, $p_{3}$ такие, что верны следующие равенства. }
	\end{frame}
	
	\begin{frame}{Часть 1: Общий подход к трехточечной аппроксимации}
		\begin{proposition}\label{pr1}
			Пусть верно 
			\begin{equation*}
				\begin{pmatrix} 
					1&1&1\\ 
					\hat{x}_{\pi_{1}}~~ &  \hat{x}_{\pi_{2}}~~  & \hat{x}_{\pi_{3}} \\ 
					\hat{x}_{\pi_{1}}^{2}~~&\hat{x}_{\pi_{2}}^{2}~~  &\hat{x}_{\pi_{3}}^{2}
				\end{pmatrix}
				\begin{pmatrix}p_{1}\\p_{2}\\ p_{3}\end{pmatrix}= \begin{pmatrix}1\\0\\1 \end{pmatrix},\label{4}
			\end{equation*}
			где $\hat{x}_{\pi_{i}} = \hat{\F}^{-1}(\pi_{i})$, $\hat{\F}(y)$ "--- функция распределения $\displaystyle{\hat{\xi} = \frac{\xi-m}{s}}$. Тогда $m=\tilde{m}$ и $s^{2} = \tilde{s}^{2}$.
		\end{proposition}
		
		\note{Пусть верна следующая система, тогда $m=\tilde{m}$ и $s^{2} = \tilde{s}^{2}$. Это Предложение дает требуемую аппроксимацию дискретным распределением, если найденные вероятности $p_{i}$ являются неотрицательными. }
		
		%\bigskip
	\end{frame}
	
	\begin{frame}{Часть 2: Аппроксимация нормального распределения}
		\begin{equation*}
			\begin{pmatrix} 1&1&1\\ 
				\Phi^{-1}(\pi)~~ &  0~~  & -\Phi ^{-1}(\pi) \\ 
				\Phi ^{-1}(\pi)^{2}~~& 0~~  &\Phi ^{-1}(\pi)^{2}
			\end{pmatrix} 
			\begin{pmatrix}p_{\pi}\\p_{0.5}\\ p_{1-\pi}\end{pmatrix}= \begin{pmatrix}1\\0\\1\end{pmatrix}.
		\end{equation*}
		
		\begin{proposition}\label{pr7}
			$\xi\sim N(\mu, \sigma)$, пусть верно 
			\begin{equation*}
				\begin{cases}
					p_{\pi} = p_{1-\pi}=\displaystyle{\frac{\delta}{2}},\\ 
					p_{0.5}=1-\delta,
				\end{cases}\label{7}
			\end{equation*}
			где $\delta  = \displaystyle{\frac{1}{\Phi ^{-1}(\pi)^{2}}}$. Тогда $m=\tilde{m}$ и $s^{2} = \tilde{s}^{2}$.
		\end{proposition}
		
		\note{Если $\xi\sim N(\mu, \sigma) $ имеет нормальное распределение, то
			$\hat{\xi}$ имеет стандартное нормальное распределение, поэтому $\hat{\xi}\sim N(0, 1)$ в предыдущем Предложении. Поэтому значения вероятностей можно выразить через функцию распределения стандартного нормального распределения.}
	\end{frame}
	
	\begin{frame}{Часть 2: Аппроксимация нормального распределения}
		Рассмотрим частный случай $\pi = 0.1$, имеем 
		$$\Phi ^{-1}(0.1) = -\Phi ^{-1}(0.9) \approx  -1.28, \qquad \Phi ^{-1}(0.5) = 0. $$
		\begin{equation*}
			\begin{cases}
				p_{1}\approx 0.305, \\ 
				p_{2}\approx 0.390,  \\ 
				p_{3}\approx 0.305.
			\end{cases}
		\end{equation*}
		
		Эти вероятности примерно равны 0.3, 0.4, 0.3, поэтому это правило называют \textcolor{blue}{\hbox{\textbf{правилом 30-40-30}}}.
		
		\note{Рассматриваем частный случай $\pi = 0.1$. Для него вероятности примерно равны 0.3, 0.4, 0.3, поэтому это правило называют правилом 30-40-30.}
		
	\end{frame}	
	
	\begin{frame}{Часть 3: Аппроксимация логнормального распределения}
		\begin{pr1}
			квантили $x_{\pi_{1}}, x_{\pi_{2}}, x_{\pi_{3}}$ логнормальной случайной величины $\eta$, $\ln(\eta) \sim N(\mu, \sigma)$.
		\end{pr1}
		
		\bigskip
		
		\begin{enumerate}
			\item Вычисляем значения мат. ожидания $m$ и дисперсии $s^{2}$ случайной величины $\eta$, используя известные $x_{\pi_{1}}, x_{\pi_{2}}, x_{\pi_{3}}$.
			\item Выражаем параметры $\mu$ и $\sigma$ мат. ожидание и дисперсию соответствующего нормального распределения через параметры $m$ и $s^{2}$ логнормального распределения
			\item С помощью системы уравнений из метода для нормального распределения находим значения весов $p_{1}$, $p_{2}$, $p_{3}$.
		\end{enumerate}
		\bigskip
		\begin{pr3}\end{pr3} веса $p_{1}$, $p_{2}$, $p_{3}$ для $x_{\pi_{1}}, x_{\pi_{2}}, x_{\pi_{3}}$ случайной величины $\tilde{\xi}$.
		
		\note{Построим алгоритм для нахождения трехточечной симметричной аппроксимации логнормального распределения. Заданы три квантиля $x_{\pi_{1}}, x_{\pi_{2}}, x_{\pi_{3}}$, вычисляем по ним мат. ожидание $m$ и дисперсию $s^{2}$ случайной величины $\eta$. Затем выражаем параметры $\mu$ и $\sigma$ и наконец находим значения весов $p_{1}$, $p_{2}$, $p_{3}$.}
	\end{frame}
	
	\begin{frame}{Часть 3: Условие на параметр $\sigma$}
		Мною доказаны следующие предложения.
		\begin{proposition2}
			Неотрицательные вероятности $p_{1}$, $p_{2}$, $p_{3}$ для аппроксимации логнормальной случайной величины $\eta$ существуют только при условии \[\exp(\sigma^{2})+\exp(-\sigma^{2})-\exp\left( -\dfrac{\sigma^{2}}{2}\right) 
			(\exp(c\sigma)+\exp(-c\sigma))\leq 0,\] 
			где $c = \Phi^{-1}(\pi)$.
			%(при $\sigma\leq 0.6913$, $\gamma_{3}\leq 2.8278$ примерно).
		\end{proposition2}
		
		\begin{proposition2}
			При уменьшении значения $\pi$ диапазон значений $\sigma$ увеличивается.
		\end{proposition2}
		
		\note{Мы рассмотрели способ вычисления весов для квантилей при аппроксимации логнормального распределения. Но найденные веса являются вероятностями не при любом $\sigma$. Выясним, какое должно быть ограничение на этот параметр. В следующем предложении получено условие на $\sigma$ для существования трехточечной симметричной аппроксимации логнормального распределения. При уменьшении значения $\pi$ диапазон значений $\sigma$ увеличивается.}
		
	\end{frame}
	
	\begin{frame}{Часть 3: Варианты постановки задачи}
		\textbf{Задача:} имеются квантили $x_{\pi}$, $x_{0.5}$, $x_{1-\pi}$ логнормальной случайной величины $\eta$. Нужно уметь считать $m$ и $s^{2}$.
		
		\begin{enumerate}
			\item Используя значения двух квантилей, найти значения параметров $\mu$ и $\sigma$ нормальной случайной величины $\ln(\eta)\sim N(\mu, \sigma)$. Через них вычислить значения $m$ и $s^{2}$.
			\item Найти значения $p_{1}$, $p_{2}$, $p_{3}$ такие, что 
			\begin{equation*}
				p_{1}x_{\pi_{1}} + p_{2}x_{\pi_{2}} + p_{3}x_{\pi_{3}} = m, \label{2}
			\end{equation*}
			\begin{equation*}
				p_{1} x_{\pi_{1}}^{2} + p_{2} x_{\pi_{2}}^{2} + p_{3} x_{\pi_{3}}^{2} - m^{2} = s^{2}.
			\end{equation*}
			
		\end{enumerate}
		
		\note{Рассмотрим варианты постановки задачи поиска $m$ и $s^{2}$. Можно не переходить к аппроксимации дискретным распределением, а сразу вычислить моменты, используя значения квантилей. Второй вариант искать значения $p_{1}$, $p_{2}$, $p_{3}$. Если они положительные, то рассматривается аппроксимация дискретной $\tilde{\xi}$, у которой $\tilde{m} = m$ и $\tilde{s}^{2}=s^{2}$.
			Если не все положительные, то можно воспринимать задачу формально, как поиск коэффициентов линейной комбинации $x_{\pi}$, $x_{0.5}$, $x_{1-\pi}$. }
	\end{frame}
	
	\begin{frame}{Часть 3: Точность неправильной аппроксимации}
		\textbf{Проблема:} метод Свонсона выведенный для аппроксимации нормального распределения используют для логнормального.
		
		\textbf{Вопрос:} какова точность аппроксимации $m$ и $s^{2}$?
		
		\begin{proposition2}\label{pr5}
			Ошибка аппроксимации мат.ожидания логнормального распределения по методу Свонсона, выведенному для аппроксимации нормального распределения, равна
			\[\dfrac{\mid m - \widetilde{m} \mid}{m} = \biggl| \exp\left( \dfrac{\sigma^{2}}{2}\right)  - \dfrac{1}{2 c^{2}}\times\]\[\times(\exp(c\sigma)-1 +\exp(-c\sigma)) + 1 \biggr|/\exp\left(\dfrac{\sigma^{2}}{2}\right),\]
			где $c = \Phi^{-1}(\pi)$, и не зависит от параметра $\mu$.
		\end{proposition2}
		
		\note{Предлагаемые методы аппроксимации трехточечным дискретным распределением логнормального распределения не работают при $\sigma \leq 0.6913$. На практике часто используют метод Свонсона выведенный для аппроксимации нормального распределения. Мною доказано это предложение. Ошибка аппроксимации мат. ожидания выражается следующим образом и не зависит от параметра $\mu$. }
	\end{frame}
	
	\begin{frame}{Часть 3: Точность метода Свонсона для логнормального распределения}
		\begin{proposition2}\label{pr6}
			Ошибка аппроксимации дисперсии логнормального распределения по методу Свонсона, выведенному для аппроксимации нормального распределения, равна
			\[\dfrac{\mid s^{2} - \widetilde{s}^{2} \mid}{s^{2}} = \biggl|\exp(\sigma^{2})(\exp(\sigma^{2}-1)) -\]\[- \dfrac{1}{2c^{2}}\exp(-2c\sigma)- \left( 1- \dfrac{1}{c^{2}}\right) \exp(2c\sigma)+\]\[+ \left( \dfrac{1}{2c^{2}}(\exp(c\sigma)-1+\exp(-c\sigma)) + 1\right) ^{2}\biggr| /\exp(\sigma^{2})(\exp(\sigma^{2}-1)),\]
			где $c = \Phi^{-1}(\pi)$, и не зависит от параметра $\mu$.
			
			%\mid
			
		\end{proposition2}
		
		\note{Ошибка аппроксимации дисперсии выражается следующим образом и тоже не зависит от параметра $\mu$.}
	\end{frame}
	
	\begin{frame}{Часть 3: Точность метода Свонсона для логнормального распределения}
		
		\begin{figure}[h]
			\begin{center}
				\begin{minipage}[h]{0.6\linewidth}
					\includegraphics[width=1\linewidth]{img/par_new2.jpg}
					\caption{Относительная ошибка аппроксимации мат.ож. и дисперсии} %% подпись к рисунку
					\label{ris:image2} %% метка рисунка для ссылки на него
				\end{minipage}
				
			\end{center}
		\end{figure}
		
		\note{Теперь посмотрим на график зависимости ошибок аппроксимации мат.ожидания и дисперсии от параметра $\sigma$. Видим, что при $\sigma\leq1.5$ ошибка аппроксимации мат.ожидания меньше $12\%$, а ошибка аппроксимации дисперсии может достигать $80\%$. Для интересующих нас значений $\sigma\geq 0.69$ ошибка мат. ожидания может быть как маленькой, так и очень большой. Ошибка дисперсии при этом точно больше $25\%$. }
		
		%Видим, что при $\sigma\leq1.5$ ошибка аппроксимации мат.ожидания меньше $12\%$. 
		%Видим, что при $\sigma\leq1.5$ ошибка аппроксимации дисперсии может достигать $80\%$. 
		
	\end{frame}
	
	\begin{frame}{Часть 4: Произведение двух логнормальных распределений}
		Рассмотрим две логнормальные случайные величины
		\begin{itemize}
			\item $\ln(\xi_{1}) \sim N(\mu_{1}, \sigma _{1}^{2})$,
			\item $\ln(\xi_{2}) \sim N(\mu_{2}, \sigma _{2}^{2})$,
		\end{itemize}
		которые заданы своими квантилями
		\begin{itemize}
			\item $x_{\pi}$, $x_{0.5}$, $x_{1-\pi}$ "--- симметричные квантили $\xi_{1}$,
			\item $y_{\pi}$, $y_{0.5}$, $y_{1-\pi}$ "--- симметричные квантили $\xi_{2}$.
		\end{itemize}
		
		\bigskip
		
		\textbf{Задача:} аппроксимировать непрерывную случайную величину $\eta = \xi_{1}\xi_{2}$ дискретной, то есть найти квантили вида $z_{\pi}$, $z_{0.5}$, $z_{1-\pi}$.
		
		\note{Рассмотрим произведение логнормально распределенных случайных величин. Эта процедура применяется в нефтяной промышленности. Для случайных величин $\xi_{1}$ и $\xi_{2}$ заданы наборы квантилей. Ставим задачу аппроксимировать случайную величину $\eta = \xi_{1}\xi_{2}$ дискретной, то есть найти квантили вида $z_{\pi}$, $z_{0.5}$, $z_{1-\pi}$.}
		
	\end{frame}
	
	\begin{frame}{Часть 4: Произведение двух логнормальных распределений}
		
		\begin{proposition}
			Зная квантили $x_{\pi}$, $x_{0.5}$, $x_{1-\pi}$ случайной величины $\xi_{1}$ и квантили $y_{\pi}$, $y_{0.5}$, $y_{1-\pi}$ случайной величины $\xi_{2}$ можно найти квантили $z_{\pi}$, $z_{0.5}$, $z_{1-\pi}$ случайной величины $\xi_{1}\xi_{2}$, как
			
			\begin{equation*}
				z_{\pi}=\exp(b\Phi^{-1}(\pi)+a),
			\end{equation*}
			\begin{equation*}
				z_{0.5}=x_{0.5}y_{0.5},
			\end{equation*}
			\begin{equation*}
				z_{1-\pi}=\exp(b\Phi^{-1}(1-\pi)+a),
			\end{equation*}
			
			где $a$ и $b$ такие, что прямая $y=\dfrac{x-a}{b}$, проходит через точки $(\ln(x_{\pi}y_{\pi}), t)$ и $(\ln(x_{0.5}y_{0.5}),0)$ при
			\begin{equation*}
				t = \frac{\Phi^{-1}(\pi)((\ln(x_{0.5})+\ln(y_{0.5}))-(\ln(x_{\pi})+\ln(y_{\pi})))}{\sqrt{(\ln(x_{0.5})-\ln(x_{\pi}))^{2}+(\ln(y_{0.5})-\ln(y_{\pi}))^{2}}}. 
			\end{equation*}
		\end{proposition}
		
		\note{Следующее предложение было мною подробно доказано, но сама идея доказательства взята из статьи. Зная квантили $x_{\pi}$, $x_{0.5}$, $x_{1-\pi}$ случайной величины $\xi_{1}$ и квантили $y_{\pi}$, $y_{0.5}$, $y_{1-\pi}$ случайной величины $\xi_{2}$ можно найти квантили $z_{\pi}$, $z_{0.5}$, $z_{1-\pi}$ случайной величины $\xi_{1}\xi_{2}$.}
		
	\end{frame}
	
	
	\begin{frame}{Часть 5: Сумма двух логнормальных распределений }
		
		Рассмотрим сумму двух логнормальных случайных величин.
		\begin{equation*}
			\ln(\xi_{1}) \sim N(\mu_{1}, \sigma _{1}^{2}),
		\end{equation*}
		\begin{equation*}
			\ln(\xi_{2}) \sim N(\mu_{2}, \sigma _{2}^{2}),
		\end{equation*}
		\begin{equation*}
			\xi = \xi_{1}+\xi_{2}.
		\end{equation*}
		$\xi_{1}$ и $\xi_{2}$ заданы своими квантилями.
		\bigskip
		
		Поставим задачу аппроксимации суммы логнормальным распределением $\ln(\eta)\sim N(\mu, \sigma)$, так как нужно рассматривать сумму не обязательно двух, а произвольного числа случайных величин. 
		
		\bigskip
		
		\textbf{Задача:} найти квантили $z_{\pi}$, $z_{0.5}$, $z_{1-\pi}$ случайной величины $\eta$.
		
		\note{Рассмотрим сумму двух логнормальных случайных величин,которые заданы наборами своих квантилей. Поставим задачу аппроксимации суммы логнормальным распределением $\ln(\eta)\sim N(\mu, \sigma)$, так как нужно рассматривать сумму не обязательно двух, а произвольного числа случайных величин. По известным квантилям уже знаем, как вычислить вероятности $p_{1}$, $p_{2}$, $p_{3}$ такие, что $m = \tilde{m}$  и $s^{2} = \tilde{s}^{2}$.}
		
		
	\end{frame}
	
	\begin{frame}{Часть 5: Сумма двух логнормальных распределений}
		
		\begin{pr1}
			Квантили $x_{\pi}$, $x_{0.5}$, $x_{1-\pi}$ "--- квантили $\xi_{1}$, $y_{\pi}$, $y_{0.5}$, $y_{1-\pi}$ "--- квантили $\xi_{2}$.
		\end{pr1}
		\begin{enumerate}
			\item $x_{\pi}$, $x_{0.5}$, $x_{1-\pi}$ $\rightarrow$  $\mu_{1}$, $\sigma_{1}$
			\item $y_{\pi}$, $y_{0.5}$, $y_{1-\pi}$ $\rightarrow$ $\mu_{2}$, $\sigma_{2}$
			\item $\mu_{i}$, $\sigma_{i}$ $\rightarrow$ $m_{i}$, $s_{i}^{2}$
			\item $m = m_{1}+m_{2}$
			\item $s^{2}=s_{1}^{2} + s_{2}^{2}$
			\item $m$, $s^{2}$ $\rightarrow$ $\mu$, $\sigma$
			\item $\mu$, $\sigma$ $\rightarrow$ $z_{\pi}$, $z_{0.5}$, $z_{1-\pi}$
			\item $z_{\pi}$, $z_{0.5}$, $z_{1-\pi}$ $\rightarrow$ $p_{1}$, $p_{2}$, $p_{3}$
			
		\end{enumerate}
		\begin{pr3}\end{pr3} вероятности $p_{1}$, $p_{2}$, $p_{3}$ для квантилей $z_{\pi_{1}}, z_{\pi_{2}}, z_{\pi_{3}}$ случайной величины $\xi_{1} + \xi_{2}$.
		
		\note{Получен следующий алгоритм аппроксимации суммы двух логнормальных распределений. По набору квантилей $\xi_{1}$ находим параметры $\mu_{1}$, $\sigma_{1}$ соответствующего нормального распределения. По набору квантилей $\xi_{2}$ находим параметры $\mu_{2}$, $\sigma_{2}$ соответствующего нормального распределения. Потом находим мат. ожидания и дисперсии $\xi_{1}$ и $\xi_{2}$. Затем вычисляем мат.ожидание и дисперсию $\xi_{1}+\xi_{2}$. Через них находим параметры нормального распределения и затем находим значения квантилей через $\mu$ и $\sigma$.}
		
	\end{frame}
	
	
	\begin{frame}{Часть 5: Сумма двух логнормальных распределений. Точность аппроксимации }
		
		Oшибки аппроксимации квантилей $q_{\pi}$, $q_{0.5}$, $q_{1-\pi}$ случайной величины $\xi$ равны
		\[\dfrac{\left| q_{\pi} - z_{\pi}\right|}{q_{\pi}}, \quad\quad \dfrac{\left| q_{0.5} - z_{0.5}\right|}{q_{0.5}}, \quad\quad \dfrac{\left| q_{1-\pi} - z_{1-\pi}\right|}{q_{1-\pi}},\] где
		\[z_{100p} = F_{\eta}^{-1}(p) = \exp(\mu+\sigma\sqrt{2}\mathrm{erf}^{-1}(2p-1)).\]
		Значение квантилей $q_{i}$ выражаются как $q_{100p} = F_{\xi}^{-1}(p)$, где
		\[F_{\xi}(x) = \int_{0}^{x}\left( \dfrac{1}{2}+\dfrac{1}{2} \mathrm{erf}\left( \dfrac{\ln(x-y)-\mu_{1}}{\sigma_{1}\sqrt{2}}\right) \right)\times\]
		\[\times \left( \dfrac{1}{\sqrt{2\pi}y\sigma_{2}}\exp\left( -\left( \dfrac{\ln(y)-\mu_{2}}{\sqrt{2}\sigma_{2}}\right) ^{2}\right) \right) dy.\]
		
		\note{Выразим ошибки аппроксимации квантилей $q_{\pi}$, $q_{0.5}$, $q_{1-\pi}$ случайной величины $\xi$, здесь параметры $\mu$, $\sigma$ можно найти через параметры случайных величин $\xi_{1}$, $\xi_{2}$ и вычисленные значения
			$m = \exp\left( \mu_{1}+\frac{\sigma_{1}^{2}}{2}\right) + \exp\left( \mu_{2}+\frac{\sigma_{2} ^{2}}{2}\right)$,
			$s^{2} = m_{1}^{2}(\exp(\sigma_{1}^{2})-1)+m_{2}^{2}(\exp(\sigma_{2}^{2})-1)$, так как для независимых случайных величин мат.ожидание суммы равно сумме мат.ожиданий, дисперсия суммы равна сумме дисперсий. Функция $F_{\xi}(x)$ получена по формуле свертки.}
		
	\end{frame}
	
	\begin{frame}{Часть 5: Сумма двух логнормальных распределений. Точность аппроксимации }
		Рассмотрим $\ln(\xi_{1}) \sim N(4, \sigma _{1}^{2})$, $\ln(\xi_{2}) \sim N(4, \sigma _{2}^{2})$ и найдем ошибки с помощью моделирования, объемы выборок равны $10^{6}$.
		
		%\bigskip
		
		\begin{table}[ht]
			\centering
			\caption{Ошибка аппроксимации медианы ($\%$) в зависимости от $\sigma_{1}^{2}$ (строка) и $\sigma_{2}^{2}$ (столбец)}
			\begin{tabular}{rrrrrr}
				\hline
				& \textbf{0.25} & \textbf{0.75} & \textbf{1.25} & \textbf{1.75} & \textbf{2.25} \\ 
				\hline
				\textbf{0.25} & 0.24 & 0.46 & 4.19 & 11.67 & 21.10 \\ 
				\textbf{0.75} & 0.74 & 0.40 & 3.06 & 11.46 & 20.99 \\ 
				\textbf{1.25} & 4.25 & 3.27 & 2.48 & 6.18 & 16.15 \\ 
				\textbf{1.75} & 12.18 & 10.12 & 5.57 & 5.24 & 9.92 \\ 
				\textbf{2.25} & 20.94 & 20.20 & 16.29 & 9.59 & 8.47 \\ 
				\hline
			\end{tabular}
		\end{table}
		\note{Рассмотрим $\ln(\xi_{1}) \sim N(4, \sigma _{1}^{2})$, $\ln(\xi_{2}) \sim N(4, \sigma _{2}^{2})$ и найдем ошибки с помощью моделирования, объемы выборок равны $10^{6}$. В данной таблице представлены ошибки для аппроксимации медианы. По построению аппроксимации суммы двух логнормальных распределений логнормальным распределением ошибки мат. ожидания и дисперсии равны 0, то есть $m=\tilde{m}$ и $s^{2} = \tilde{s}^{2}$. Но если для каких-либо расчетов понадобятся квантили $\eta$, то ошибка медианы может достигать 21\%. }
		
	\end{frame}
	
	\begin{frame}{Часть 5: Сумма двух логнормальных распределений. Точность аппроксимации}
		\begin{table}[ht]
			\centering
			\caption{Ошибка аппроксимации $q_{10}$ ($\%$) в зависимости от $\sigma_{1}^{2}$ (строка) и $\sigma_{2}^{2}$ (столбец)}
			\begin{tabular}{rrrrrr}
				\hline
				& \textbf{0.25} & \textbf{0.75} & \textbf{1.25} & \textbf{1.75} & \textbf{2.5} \\
				\hline
				\textbf{0.25} & 2.35 & 13.59 & 23.93 & 33.20 & 42.75 \\ 
				\textbf{0.75} & 1.20 & 10.54 & 21.80 & 33.93 & 42.82 \\ 
				\textbf{1.25} & 3.02 & 7.03 & 18.43 & 29.49 & 40.09 \\ 
				\textbf{1.75} & 14.45 & 5.27 & 14.33 & 26.50 & 36.75 \\ 
				\textbf{2.5} & 34.70 & 11.44 & 11.10 & 23.05 & 32.84 \\ 
				\hline
			\end{tabular}
		\end{table}
		
		\note{В данной таблице представлены ошибки для аппроксимации $q_{10}$, они могут достигать 67\%. }
		
	\end{frame}
	
	\begin{frame}{Часть 5: Сумма двух логнормальных распределений. Точность аппроксимации}
		
		\begin{table}[ht]
			\centering
			\caption{Ошибка аппроксимации $q_{90}$ ($\%$) в зависимости от $\sigma_{1}^{2}$ (строка) и $\sigma_{2}^{2}$ (столбец) }
			\begin{tabular}{rrrrrr}
				\hline
				& \textbf{0.25} & \textbf{0.75} & \textbf{1.25} & \textbf{1.75} & \textbf{2.25} \\ 
				\hline
				\textbf{0.25} & 0.16 & 5.03 & 13.69 & 17.74 & 19.37 \\ 
				\textbf{0.75} & 5.63 & 1.89 & 5.77 & 10.70 & 16.03 \\ 
				\textbf{1.25} & 13.55 & 5.75 & 2.52 & 6.00 & 9.79 \\ 
				\textbf{1.75} & 19.95 & 11.88 & 5.77 & 3.50 & 4.89 \\ 
				\textbf{2.25} & 18.47 & 15.44 & 9.42 & 5.50 & 5.27 \\ 
				\hline
			\end{tabular}
		\end{table}
		
		\note{В данной таблице представлены ошибки для аппроксимации $q_{90}$, они могут достигать 20\%. }
		
	\end{frame}
	
	\begin{frame}{Часть 5: Сумма двух логнормальных распределений. Точность аппроксимации}
		
		\begin{table}[]
			\caption{Коэффициент асимметрии суммы (голубой) и аппроксимации (розовый) в зависимости от $\sigma_{1}^{2}$ (строка) и $\sigma_{2}^{2}$ (столбец) }
			\begin{tabular}{|c|l|l|l|l|l|}
				\hline
				& \textbf{0.25} & \textbf{0.75} & \textbf{1.25} & \textbf{1.75} & \textbf{2.5} \\ \hline
				\multirow{2}{*}{\textbf{0.25}} & \textcolor{cyan}{1.77}          & \textcolor{cyan}{4.23}          & \textcolor{cyan}{6.71}          & \textcolor{cyan}{15.59}         & \textcolor{cyan}{16.68}        \\ \cline{2-6} 
				& \textcolor{magenta}{1.53}          & \textcolor{magenta}{3.75}          & \textcolor{magenta}{7.48}          & \textcolor{magenta}{14.76}         & \textcolor{magenta}{29.70}        \\ \hline
				\multirow{2}{*}{\textbf{0.75}} & \textcolor{cyan}{1.66}          & \textcolor{cyan}{3.86}          & \textcolor{cyan}{7.39}          & \textcolor{cyan}{11.43}         & \textcolor{cyan}{54.43}        \\ \cline{2-6} 
				& \textcolor{magenta}{1.55}          & \textcolor{magenta}{3.65}          & \textcolor{magenta}{7.22}          & \textcolor{magenta}{14.25}         & \textcolor{magenta}{28.77}        \\ \hline
				\multirow{2}{*}{\textbf{1.25}} & \textcolor{cyan}{2.13}          & \textcolor{cyan}{3.68}          & \textcolor{cyan}{8.73}          & \textcolor{cyan}{13.76}         & \textcolor{cyan}{29.28}        \\ \cline{2-6} 
				& \textcolor{magenta}{1.71}          & \textcolor{magenta}{3.60}          & \textcolor{magenta}{6.97}          & \textcolor{magenta}{13.68}         & \textcolor{magenta}{27.66}        \\ \hline
				\multirow{2}{*}{\textbf{1.75}} & \textcolor{cyan}{5.88}          & \textcolor{cyan}{4.06}          & \textcolor{cyan}{7.50}          & \textcolor{cyan}{31.50}         & \textcolor{cyan}{24.89}        \\ \cline{2-6} 
				& \textcolor{magenta}{2.17}          & \textcolor{magenta}{3.71}          & \textcolor{magenta}{6.79}          & \textcolor{magenta}{13.09}         & \textcolor{magenta}{26.41}        \\ \hline
				\multirow{2}{*}{\textbf{2.5}}  & \textcolor{cyan}{11.18}         & \textcolor{cyan}{8.85}          & \textcolor{cyan}{8.55}          & \textcolor{cyan}{10.34}         & \textcolor{cyan}{23.61}        \\ \cline{2-6} 
				& \textcolor{magenta}{3.30}          & \textcolor{magenta}{4.29}          & \textcolor{magenta}{6.90}          & \textcolor{magenta}{12.66}         & \textcolor{magenta}{25.13}        \\ \hline
			\end{tabular}
		\end{table}
		
		\note{Теперь посмотрим на таблицу с коэффициентами асимметрии для суммы $\xi_{1}$+$\xi_{2}$, они выделены голубым цветом и на коэффициенты асимметрии для аппроксимации, они выделены розовым цветом. }
	\end{frame}
	
	\begin{frame}{Часть 5: Сумма двух логнормальных распределений. Точность аппроксимации}
		
		\begin{table}[]
			\centering
			\caption{Коэффициент эксцесса суммы (голубой) и аппроксимации (розовый) в зависимости от $\sigma_{1}^{2}$ (строка) и $\sigma_{2}^{2}$ (столбец) }
			\begin{tabular}{|c|l|l|l|l|l|}
				\hline
				& \textbf{0.25} & \textbf{0.75} & \textbf{1.25} & \textbf{1.75} & \textbf{2.5} \\ \hline
				\multirow{2}{*}{\textbf{0.25}} & \textcolor{cyan}{6.54}          & \textcolor{cyan}{51.70}         & \textcolor{cyan}{227.68}        & \textcolor{cyan}{408.58}        & \textcolor{cyan}{734.47}       \\ \cline{2-6} 
				& \textcolor{magenta}{4.42}          & \textcolor{magenta}{32.60}         & \textcolor{magenta}{180.39}        & \textcolor{magenta}{1088.57}       & \textcolor{magenta}{7274.56}      \\ \hline
				\multirow{2}{*}{\textbf{0.75}} & \textcolor{cyan}{6.21}          & \textcolor{cyan}{61.66}         & \textcolor{cyan}{144.59}        & \textcolor{cyan}{201.69}        & \textcolor{cyan}{1304.88}      \\ \cline{2-6} 
				& \textcolor{magenta}{4.56}          & \textcolor{magenta}{30.53}         & \textcolor{magenta}{164.86}        & \textcolor{magenta}{990.42}        & \textcolor{magenta}{6666.16}      \\ \hline
				\multirow{2}{*}{\textbf{1.25}} & \textcolor{cyan}{11.47}         & \textcolor{cyan}{27.75}         & \textcolor{cyan}{179.22}        & \textcolor{cyan}{193.95}        & \textcolor{cyan}{546.57}       \\ \cline{2-6} 
				& \textcolor{magenta}{5.61}          & \textcolor{magenta}{29.53}         & \textcolor{magenta}{150.21}        & \textcolor{magenta}{886.71}        & \textcolor{magenta}{5989.44}      \\ \hline
				\multirow{2}{*}{\textbf{1.75}} & \textcolor{cyan}{122.65}        & \textcolor{cyan}{46.01}         & \textcolor{cyan}{110.03}        & \textcolor{cyan}{276.24}        & \textcolor{cyan}{14081.05}     \\ \cline{2-6} 
				& \textcolor{magenta}{9.44}          & \textcolor{magenta}{31.88}         & \textcolor{magenta}{140.69}        & \textcolor{magenta}{788.78}        & \textcolor{magenta}{5280.07}      \\ \hline
				\multirow{2}{*}{\textbf{2.5}}  & \textcolor{cyan}{195.77}        & \textcolor{cyan}{283.81}        & \textcolor{cyan}{344.56}        & \textcolor{cyan}{4837.85}       & \textcolor{cyan}{1292.23}      \\ \cline{2-6} 
				& \textcolor{magenta}{24.08}         & \textcolor{magenta}{44.88}         & \textcolor{magenta}{146.68}        & \textcolor{magenta}{720.26}        & \textcolor{magenta}{4612.33}      \\ \hline
			\end{tabular}
		\end{table}
		
		\note{Также можно посмотреть на коэффициенты эксцесса суммы и аппроксимации.}
	\end{frame}
	
	
	\begin{frame}{Часть 5: Сумма двух логнормальных распределений. Точность аппроксимации}
		
		Посчитаем значения функции $F_{\xi}(x)$ от квантилей $z_{10}$, $z_{50}$, $z_{90}$ случайной величины $\eta$. Они показывают, каким квантилем для $\xi$ являются квантили $z_{i}$. Результаты приведены в следующих таблицаx.
		
		\begin{table}[!hhh]
			\centering
			\caption{$F_{\eta}(z_{50})$ ($\%$) в зависимости от $\sigma_{1}^{2}$ (строка) и $\sigma_{2}^{2}$ (столбец) }
			\label{tab4}
			\begin{tabular}{rrrrrr}
				\hline
				& \textbf{0.25} & \textbf{0.75} & \textbf{1.25} & \textbf{1.75} & \textbf{2.25} \\ 
				\hline
				\textbf{0.25} & 50.10 & 49.87 & 46.96 & 42.18 & 36.95 \\ 
				\textbf{0.75} & 49.87 & 49.82 & 48.30 & 44.55 & 39.74 \\ 
				\textbf{1.25} & 46.96 & 48.30 & 49.00 & 47.19 & 43.31 \\ 
				\textbf{1.75} & 42.18 & 44.55 & 47.19 & 47.91 & 46.03 \\ 
				\textbf{2.25} & 36.95 & 39.74 & 43.31 & 46.03 & 46.73 \\ 
				\hline
			\end{tabular}
		\end{table}
		
		\note{Посчитаем значения функции $F_{\xi}(x)$ от квантилей $z_{10}$, $z_{50}$, $z_{90}$ случайной величины $\eta$. Они показывают, каким квантилем для $\xi$ являются квантили $z_{i}$. В данной таблице приведены значения $F_{\eta}(z_{50})$. Видим, что они не сильно далеки от нужных 50\%.}
	\end{frame}
	
	\begin{frame}{Часть 5: Сумма двух логнормальных распределений. Точность аппроксимации}
		
		\begin{table}[!hhh]
			\centering
			\caption{$F_{\eta}(z_{10})$ ($\%$) в зависимости от $\sigma_{1}^{2}$ (строка) и $\sigma_{2}^{2}$ (столбец) }
			\label{tab5}
			\begin{tabular}{rrrrrr}
				\hline
				& \textbf{0.25} & \textbf{0.75} & \textbf{1.25} & \textbf{1.75} & \textbf{2.25} \\ 
				\hline
				\textbf{0.25} & 9.79 & 5.84 & 1.82 & 0.32 & 0.04 \\ 
				\textbf{0.75} & 5.84 & 8.89 & 6.45 & 3.14 & 1.19 \\ 
				\textbf{1.25} & 1.82 & 6.45 & 7.85 & 6.00 & 3.35 \\ 
				\textbf{1.75} & 0.32 & 3.14 & 6.00 & 6.89 & 5.43 \\ 
				\textbf{2.25} & 0.04 & 1.19 & 3.35 & 5.43 & 6.08 \\ 
				\hline
			\end{tabular}
		\end{table}
		
		\note{В данной таблице приведены значения $F_{\eta}(z_{10})$.}
	\end{frame}
	
	\begin{frame}{Часть 5: Сумма двух логнормальных распределений. Точность аппроксимации}
		
		\begin{table}[!hhh]
			\centering
			\caption{$F_{\eta}(z_{90})$ ($\%$) в зависимости от $\sigma_{1}^{2}$ (строка) и $\sigma_{2}^{2}$ (столбец)}
			\label{tab6}
			\begin{tabular}{rrrrrr}
				\hline
				& \textbf{0.25} & \textbf{0.75} & \textbf{1.25} & \textbf{1.75} & \textbf{2.25} \\
				\hline
				\textbf{0.25} & 90.08 & 91.47 & 92.31 & 92.42 & 92.19 \\ 
				\textbf{0.75} & 91.47 & 90.38 & 91.11 & 91.83 & 92.02 \\ 
				\textbf{1.25} & 92.31 & 91.11 & 90.57 & 90.93 & 91.31 \\ 
				\textbf{1.75} & 92.42 & 91.83 & 90.93 & 90.62 & 90.75 \\ 
				\textbf{2.25} & 92.19 & 92.02 & 91.31 & 90.75 & 90.56 \\ 
				\hline
			\end{tabular}
		\end{table}
		
		\note{В данной таблице приведены значения $F_{\eta}(z_{90})$.}
	\end{frame}
	
	\begin{frame}{Часть 5: Сумма двух логнормальных распределений. Точность аппроксимации}
		
		Построим оценки плотности для $\xi$ и $\eta$, когда ошибки имеют очень маленькие значения и когда достаточно большие.
		
		\begin{figure}[h]
			\begin{center}
				\begin{minipage}[h]{0.4\linewidth}
					\includegraphics[width=1\linewidth]{img/sr1.pdf}
					\caption{$\sigma_{1}^{2} = 0.25$, $\sigma_{2}^{2} = 0.25$, $err_{med} = 0.17\%$,  $err_{q_{10}} = 0.35\%$,  $err_{q_{90}} = 0.12\%$. } %% подпись к рисунку
					\label{ris7} %% метка рисунка для ссылки на него
				\end{minipage}
				
			\end{center}
		\end{figure}
		
		\note{Построим оценки плотности для $\xi$ и $\eta$, когда ошибки имеют очень маленькие значения и когда достаточно большие. На данном графике представлен случай $\sigma_{1}^{2} = 0.25$, $\sigma_{2}^{2} = 0.25$, при этом ошибки $q_{10}$, $q_{50}$, $q_{90}$ близки к нулю.}
		
	\end{frame}
	
	\begin{frame}{Часть 5: Сумма двух логнормальных распределений. Точность аппроксимации }
		
		\begin{figure}[h]
			\begin{center}
				\begin{minipage}[h]{0.53\linewidth}
					\includegraphics[width=1\linewidth]{img/sr2.pdf}
					\caption{$\sigma_{1}^{2} = 2.25$, $\sigma_{2}^{2} = 0.75$, $err_{med} = 20.4\%$,  $err_{q_{10}} = 54.13\%$,  $err_{q_{90}} = 15.54\%$. } %% подпись к рисунку
					\label{ris7} %% метка рисунка для ссылки на него
				\end{minipage}
				
			\end{center}
		\end{figure}
		
		\note{Теперь построим оценки плотности для $\xi$ и $\eta$, когда ошибки имеют большие значения. На данном графике представлен случай $\sigma_{1}^{2} = 2.25$, $\sigma_{2}^{2} = 0.75$.}
	\end{frame}
	
	\begin{frame}{Заключение}
		
		\textcolor{blue}{\hbox{\textbf{Мною были получены следующие результаты:}}}
		\begin{enumerate}
			\item Получено условие на $\sigma$ для существования трехточечной симметричной аппроксимации логнормального распределения.
			\item Численно оценена точность аппроксимации мат. ожидания и дисперсии логнормального распределения с помощью метода Свонсона, применяемого к нормальному распределению.
			\item Построен алгоритм для нахождения трехточечной симметричной аппроксимации суммы логнормальных распределений.
			\item Численно оценена точность трехточечной симметричной аппроксимации суммы логнормальных распределений.
		\end{enumerate}
		%\bigskip
		
		\note{Таким образом, мною были получены следующие результаты. 
			Получено условие на $\sigma$ для существования трехточечной симметричной аппроксимации логнормального распределения.
			Численно оценена точность аппроксимации мат. ожидания и дисперсии логнормального распределения с помощью метода Свонсона, применяемого к нормальному распределению.
			Построен алгоритм для нахождения трехточечной симметричной аппроксимации суммы логнормальных распределений.
			Численно оценена точность трехточечной аппроксимации суммы логнормальных распределений.}
		
	\end{frame}
	
	\begin{frame}{Список литературы}
		
		\begin{thebibliography}{1}
			\bibitem{Swansong} Keith G. Swanson's Swansong."--- Текст: электронный // stochastic: [сайт]."--- URL: https://www.stochastic.dk/post/swanson-s-swansong (дата обращения: 10.05.2023).
			
			\bibitem{Discretization} Bickel, J. Eric, Lake, Larry W., and John Lehman. "Discretization, Simulation, and Swanson's (Inaccurate) Mean." SPE Econ Mgmt 3 (2011): 128–140. doi: https://doi.org/10.2118/148542-PA.
			
			\bibitem{Simulation} Bickel, J. Eric. "Discretization, Simulation, and the Value of Information." Paper presented at the SPE Annual Technical Conference and Exhibition, Denver, Colorado, USA, October 2011. doi: https://doi.org/10.2118/145690-MS.
			
		\end{thebibliography}
		%\bigskip
		
		\note{Список литературы.}
		
	\end{frame}
	
	
